\documentclass[10pt, letterpaper]{article}



% Packages:
\usepackage[
    ignoreheadfoot, % set margins without considering header and footer
    top=2 cm, % seperation between body and page edge from the top
    bottom=2 cm, % seperation between body and page edge from the bottom
    left=2 cm, % seperation between body and page edge from the left
    right=2 cm, % seperation between body and page edge from the right
    footskip=1.0 cm, % seperation between body and footer
    % showframe % for debugging 
]{geometry} % for adjusting page geometry
\usepackage{titlesec} % for customizing section titles
\usepackage{tabularx} % for making tables with fixed width columns
\usepackage{array} % tabularx requires this
\usepackage[dvipsnames]{xcolor} % for coloring text
\definecolor{primaryColor}{RGB}{0, 0, 0} % define primary color
\usepackage{enumitem} % for customizing lists
\usepackage{fontawesome5} % for using icons
\usepackage{amsmath} % for math
\usepackage[
    pdftitle={John Doe's CV},
    pdfauthor={John Doe},
    pdfcreator={LaTeX with RenderCV},
    colorlinks=true,
    urlcolor=primaryColor
]{hyperref} % for links, metadata and bookmarks
\usepackage[pscoord]{eso-pic} % for floating text on the page
\usepackage{calc} % for calculating lengths
\usepackage{bookmark} % for bookmarks
\usepackage{lastpage} % for getting the total number of pages
\usepackage{changepage} % for one column entries (adjustwidth environment)
\usepackage{paracol} % for two and three column entries
\usepackage{ifthen} % for conditional statements
\usepackage{needspace} % for avoiding page brake right after the section title
\usepackage{iftex} % check if engine is pdflatex, xetex or luatex

% Ensure that generate pdf is machine readable/ATS parsable:
\ifPDFTeX
    \input{glyphtounicode}
    \pdfgentounicode=1
    \usepackage[T1]{fontenc}
    \usepackage[utf8]{inputenc}
    \usepackage{lmodern}
\fi

\usepackage{charter}

% Some settings:
\raggedright
\AtBeginEnvironment{adjustwidth}{\partopsep0pt} % remove space before adjustwidth environment
\pagestyle{empty} % no header or footer
\setcounter{secnumdepth}{0} % no section numbering
\setlength{\parindent}{0pt} % no indentation
\setlength{\topskip}{0pt} % no top skip
\setlength{\columnsep}{0.15cm} % set column seperation
\pagenumbering{gobble} % no page numbering

\titleformat{\section}{\needspace{4\baselineskip}\bfseries\large}{}{0pt}{}[\vspace{1pt}\titlerule]

\titlespacing{\section}{
    % left space:
    -1pt
}{
    % top space:
    0.3 cm
}{
    % bottom space:
    0.2 cm
} % section title spacing

\renewcommand\labelitemi{$\vcenter{\hbox{\small$\bullet$}}$} % custom bullet points
\newenvironment{highlights}{
    \begin{itemize}[
        topsep=0.10 cm,
        parsep=0.10 cm,
        partopsep=0pt,
        itemsep=0pt,
        leftmargin=0 cm + 10pt
    ]
}{
    \end{itemize}
} % new environment for highlights


\newenvironment{highlightsforbulletentries}{
    \begin{itemize}[
        topsep=0.10 cm,
        parsep=0.10 cm,
        partopsep=0pt,
        itemsep=0pt,
        leftmargin=10pt
    ]
}{
    \end{itemize}
} % new environment for highlights for bullet entries

\newenvironment{onecolentry}{
    \begin{adjustwidth}{
        0 cm + 0.00001 cm
    }{
        0 cm + 0.00001 cm
    }
}{
    \end{adjustwidth}
} % new environment for one column entries

\newenvironment{twocolentry}[2][]{
    \onecolentry
    \def\secondColumn{#2}
    \setcolumnwidth{\fill, 4.5 cm}
    \begin{paracol}{2}
}{
    \switchcolumn \raggedleft \secondColumn
    \end{paracol}
    \endonecolentry
} % new environment for two column entries

\newenvironment{threecolentry}[3][]{
    \onecolentry
    \def\thirdColumn{#3}
    \setcolumnwidth{, \fill, 4.5 cm}
    \begin{paracol}{3}
    {\raggedright #2} \switchcolumn
}{
    \switchcolumn \raggedleft \thirdColumn
    \end{paracol}
    \endonecolentry
} % new environment for three column entries

\newenvironment{header}{
    \setlength{\topsep}{0pt}\par\kern\topsep\centering\linespread{1.5}
}{
    \par\kern\topsep
} % new environment for the header

\newcommand{\placelastupdatedtext}{% \placetextbox{<horizontal pos>}{<vertical pos>}{<stuff>}
  \AddToShipoutPictureFG*{% Add <stuff> to current page foreground
    \put(
        \LenToUnit{\paperwidth-2 cm-0 cm+0.05cm},
        \LenToUnit{\paperheight-1.0 cm}
    ){\vtop{{\null}\makebox[0pt][c]{
        \small\color{gray}\textit{Last updated in September 2024}\hspace{\widthof{Last updated in September 2024}}
    }}}%
  }%
}%

% save the original href command in a new command:
\let\hrefWithoutArrow\href


\begin{document}
\begin{header}
    \fontsize{25 pt}{25 pt}\selectfont \VAR{candidate.name}

    \vspace{5 pt}

    \normalsize
    \mbox{\VAR{candidate.location}}%
    \kern 5.0 pt%
    |
    \kern 5.0 pt%
    \mbox{\hrefWithoutArrow{mailto:\VAR{candidate.email}}{\VAR{candidate.email}}}%
    \kern 5.0 pt%
    |
    \kern 5.0 pt%
    \mbox{\hrefWithoutArrow{tel:\VAR{candidate.phone|replace('-', '')}}{\VAR{candidate.phone}}}%
    \kern 5.0 pt%
    |
    \kern 5.0 pt%
    \mbox{\hrefWithoutArrow{\VAR{candidate.website}}{\VAR{candidate.website}}}%
    \kern 5.0 pt%
    |
    \kern 5.0 pt%
    \mbox{\hrefWithoutArrow{\VAR{candidate.linkedin}}{\VAR{candidate.linkedin}}}%
\end{header}

\section{Profile}
\begin{onecolentry}
\VAR{candidate.summary}
\end{onecolentry}

\section{Education}
\BLOCK{ for edu in education }
\begin{twocolentry}{\VAR{edu.dates}}
    \textbf{\VAR{edu.university}}, \VAR{edu.degree} (GPA: \VAR{edu.gpa})
\end{twocolentry}
\vspace{0.10 cm}
\begin{onecolentry}
\begin{highlights}
\item \VAR{edu.comments}
\end{highlights}
\end{onecolentry}
\BLOCK{ endfor }

\section{Experience}
\textbf{RESEARCH \& INDUSTRY}
\BLOCK{ for exp in experience.research_and_industry }
\vspace{0.2cm}
\begin{twocolentry}{\VAR{exp.dates}}
    \textbf{\VAR{exp.title}}, \VAR{exp.organization} -- \VAR{exp.location}
\end{twocolentry}
\begin{onecolentry}
\begin{highlights}
\BLOCK{ for bullet in exp.bullets }
\item \VAR{bullet}
\BLOCK{ endfor }
\end{highlights}
\end{onecolentry}
\BLOCK{ endfor }

\textbf{TEACHING}
\BLOCK{ for teach in experience.teaching }
\vspace{0.2cm}
\begin{twocolentry}{\VAR{teach.dates}}
    \textbf{\VAR{teach.course}}, \VAR{teach.role}
\end{twocolentry}
\begin{onecolentry}
\begin{highlights}
\BLOCK{ for bullet in teach.bullets }
\item \VAR{bullet}
\BLOCK{ endfor }
\end{highlights}
\end{onecolentry}
\BLOCK{ endfor }

\section{Publications \& Talks}
\textbf{\MakeUppercase{Journal Articles \& Preprints}}
\begin{onecolentry}
\begin{highlights}
\BLOCK{ for pub in publications } \item \VAR{pub.year} -- \VAR{pub.authors}, ``\VAR{pub.title}'', \VAR{pub.arxiv}, \VAR{pub.journal}
\BLOCK{ endfor }
\end{highlights}
\end{onecolentry}

\vspace{0.2cm}
\textbf{\MakeUppercase{Talks}}
\begin{onecolentry}
\begin{highlights}
\BLOCK{ for talk in talks }\item \VAR{talk.year} -- \VAR{talk.venue}, ``\VAR{talk.title}''
\BLOCK{ endfor }
\end{highlights}
\end{onecolentry}

\section{Coursework}
\begin{onecolentry}
\begin{highlights}
\BLOCK{ for course in coursework }
    \item \VAR{course.course}
\BLOCK{ endfor }
\end{highlights}
\end{onecolentry}

\section{Leadership \& Awards}
\textbf{AWARDS}
\begin{onecolentry}
\begin{highlights}
\BLOCK{ for award in awards }\item \VAR{award}
\BLOCK{ endfor }
\end{highlights}
\end{onecolentry}

\vspace{0.2cm}
\textbf{LEADERSHIP \& ACTIVITIES}
\begin{onecolentry}
\begin{highlights}
\BLOCK{ for role in leadership }\item \VAR{role.title}, \VAR{role.org}, \VAR{role.dates}
\BLOCK{ endfor }
\end{highlights}
\end{onecolentry}

\vspace{0.2cm}
\section{Technologies}
\begin{onecolentry}
\textbf{Languages:} \VAR{technologies.languages}
\end{onecolentry}
\vspace{0.2 cm}
\begin{onecolentry}
\textbf{Technologies:} \VAR{technologies.stack}
\end{onecolentry}
\vspace{0.2 cm}
\begin{onecolentry}
\textbf{Platforms:} \VAR{technologies.platforms}
\end{onecolentry}

\end{document}
